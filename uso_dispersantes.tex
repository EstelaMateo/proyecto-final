\documentclass{article}
\usepackage{xcolor}
\usepackage[normalem]{ulem}
\usepackage{soul}
\usepackage[a4paper, margin=2cm]{geometry}
\sethlcolor{pink}
\title{USO DE DISPERSANTES}
\author{ECOMODELAB}
\date{\today}
\begin{document}
\section{INFORMACIÓN RELEVANTE}


\hl{¿Qué dice la EPA del uso de dispersantes?}

\textbf{300.965 Mandatory Product Disclaimer.The listing of a product on the NCP Product Schedule does not constitute approval or recommendation of the product. To avoid possible misinterpretation or misrepresentation, any label, advertisement, or technical literature for products listed on the NCP Product Schedule must display in its entirety the disclaimer shown below. The disclaimer must be conspicuous and must be fully reproduced on all product literatures, labels, and electronic media including website pages.[1]}

\textbf{“Disclaimer:[PRODUCT NAME] is listed on the National Contingency Plan (NCP) Product Schedule. This listing does NOT mean that EPA approves, recommends, licenses, or certifies the use of [PRODUCT NAME] on an oil discharge. This listing means only that data have been submitted to EPA as required by Subpart J of the NCP. Only a Federal On-Scene Coordinator (OSC) may authorize use of this product in accordance with Subpart J of the NCP in response to an oil discharge.”[1]}

La inclusión de los dispersantes (4 nombrados por la EPA) en esta lista no significa que la EPA apruebe, recomiende, autorice o certifique el uso de estos productos para derrames de petróleo. Solo un “Federal On-Scene Coordinator (OSC)” puede autorizar el uso de estos productos de acuerdo con la Subpart J del NCP (National Contingency Plan), lo que vendría a ser en México la Autoridad coordinadora designada según el tipo de emergencia (SEMAR / PROFEPA / Protección Civil).

\\
\hl{¿Cuáles son los 4 dispersantes del NCP Product Schedule?}

Entre los productos de la lista entre los cuales se encuentran, “Surface Washing Agent-Agente limpiador de superficies”, “Miscellaneous Oil Spill Control Agent – MOSCA-Agente misceláneo para el control de derrames de petróleo – MOSCA”, “Surface Collection Agent-Agente colector de superficial” y los dispersantes dentro de los cuales se encuentran los siguientes [2]:
\begin{itemize}
\item ACCELL CLEAN® DWD 2.0
\item DASIC ECOSAFE OSD
\item FINASOL® OSR 52
\item FINASOL OSR 52 IBC
\end{itemize}

\hl{¿Cuál es el propósito del Substrato J del Plan Nacional de Contingencia (NCP)?} 

El objetivo del Plan Nacional de Contingencia contra la Contaminación por Hidrocarburos y Sustancias Peligrosas (NCP) es proporcionar la estructura organizativa y los procedimientos necesarios para prepararse y responder ante vertidos de hidrocarburos y emisiones de sustancias peligrosas, contaminantes y contaminantes.

\hl{¿Cuales son los requisitos para el uso de agentes dispersantes?} 



\textbf{Código de Regulaciones Federales de Estados Unidos- NATIONAL OIL AND HAZARDOUS SUBSTANCES POLLUTION CONTINGENCY PLAN.}

\textbf{Referencias}

[1] National Oil and Hazardous Substances Pollution Contingency Plan, 40 C.F.R. pt. 300. U.S. Government Publishing Office, e-CFR, accessed Feb. 15, 2026. [Online]. Available: \url{https://www.ecfr.gov/current/title-40/chapter-I/subchapter-J/part-300}

[2] U.S. Environmental Protection Agency, “Alphabetical List of NCP Product Schedule (Products Available for Use During an Oil Spill),” EPA, Dec. 15, 2025. Accessed: Feb. 15, 2026. [Online]. Available: \url{https://www.epa.gov/emergency-response/alphabetical-list-ncp-product-schedule-products-available-use-during-oil-spill#product_summaries}



\hl{Texto resaltado}

\textcolor{blue}{Este texto está en azul.}






\end{document}